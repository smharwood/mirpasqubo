\documentclass[11pt]{article}

\usepackage{amsmath,amssymb,xfrac}
\usepackage{algorithm,algorithmic}
\usepackage[margin=1in]{geometry}
\usepackage{url}

\usepackage{amsthm}
\newtheorem{theorem}{Theorem}
\newtheorem{lemma}{Lemma}
\newtheorem{proposition}{Proposition}
\newtheorem{corollary}{Corollary}
\newtheorem{assumption}{Assumption}
\theoremstyle{definition}
\newtheorem{definition}{Definition}

\newcommand{\card}[1]{\left| #1 \right|}

\newcommand{\mbf}{\mathbf}
\newcommand{\mbb}{\mathbb}
\newcommand{\mbs}{\boldsymbol}
\newcommand{\st}{\mathrm{s.t.}\;}
\newcommand{\tr}{^{\mathrm{T}}}
\newcommand{\abs}[1]{\left| #1 \right|}
\newcommand{\norm}[1]{\left\| #1 \right\|}
\newcommand{\set}[1]{\left\{ #1 \right\}}
\newcommand{\grad}{\nabla}
\newcommand{\smallsum}{{\textstyle{\sum}}}
\newcommand{\symgt}{\succ}
\newcommand{\symge}{\succeq}
\renewcommand{\hat}{\widehat}
\renewcommand{\tilde}{\widetilde}

\newcommand{\0}{\mathbf{0}}
\renewcommand{\a}{\mathbf{a}}
\renewcommand{\b}{\mathbf{b}}
\renewcommand{\c}{\mathbf{c}}
\renewcommand{\d}{\mathbf{d}}
\newcommand{\g}{\mathbf{g}}
\newcommand{\q}{\mathbf{q}}
\renewcommand{\r}{\mathbf{r}}
\newcommand{\s}{\mathbf{s}}
\renewcommand{\u}{\mathbf{u}}
\renewcommand{\v}{\mathbf{v}}
\newcommand{\w}{\mathbf{w}}
\newcommand{\x}{\mathbf{x}}
\newcommand{\y}{\mathbf{y}}
\newcommand{\z}{\mathbf{z}}

\title{Sequence-based formulation of MIRP}
%\author{Stuart M. Harwood \and Dimitar Trenev}
\date{\today}

\begin{document}
\maketitle
%%%%%%%%%%%%%%%%%%%%%%%%%%%%%%%%%%%%%%%%%%%%%
%%%%%%%%%%%%%%%%%%%%%%%%%%%%%%%%%%%%%%%%%%%%%
\section{Formulation}

Our problem is a vehicle routing problem with time windows (VRPTW).
We use a sequencing-based formulation to obtain a math program.
The setting is a graph with nodes $N \cup \set{d}$ and edges $E$.
Edges $(i,j) \in E$ have a cost $c_{i,j}$.
All nodes $n \in N$ have the same level of demand.
There is a set of vehicles $V$ available.
Assume they are homogeneous.
Each vehicle starts from the depot $d$, and we assume that the depot is ``absorbing'':
if a vehicle returns to the depot, it must remain there.
Consequently, assume $(d,d) \in E$.
The capacity of the vehicles and uniform demand levels imposes a maximum number of stops $P$ that the vehicles can make.

The variables of the math program are
\begin{itemize}
\item $x_{v,p,i} \in \set{0,1}$ : Vehicle $v$ visits node $i$ at position $p$ in its sequence ($1$) or not ($0$).
\end{itemize}


The formulation to minimize cost of travel while servicing each node is
\begin{align}
\min_x\; & \sum_v \sum_{p < P} \sum_{(i,j) \in E} c_{i,j} x_{v,p,i} x_{v,p+1,j} \\
\st
&\label{con:1}  \sum_{v \in V} \sum_{p=0}^P x_{v,p,i} = 1, \quad \forall i \in N, \\
&\label{con:2}  \sum_{i \in N \cup \set{d}} x_{v,p,i} = 1, \quad \forall v \in V, p \in \set{0,1,\dots,P}, \\
&\label{con:3}  x_{v,p,i} x_{v,p+1,j} = 0, \quad \forall v \in V, p \in \set{0,1,\dots,P-1}, (i,j) \notin E, \\
&\label{con:35} x_{v,p,d} x_{v,p+1,j} = 0, \quad \forall v \in V, p \in \set{1,\dots,P-1}, j \ne d\\
&\label{con:4}  x_{v,0,d} = 1, \quad \forall v \in V, \\
&\notag         x_{v,p,i} \in \set{0,1}, \quad \forall (v,p,i)
\end{align}

Constraint~\eqref{con:1} ensures that each node is visited exactly once over all vehicles and sequence positions
(besides the depot node $d$ -- we specify that each vehicle starts at the depot, and we allow vehicles to return to it).
Constraint~\eqref{con:2} ensures that each vehicle uses each sequence position once.
Constraint~\eqref{con:3} ensures that only allowed edges are traversed.
Constraint~\eqref{con:35} ensures that once a vehicle returns to the depot, it must remain there (cannot travel elsewhere).
Constraint~\eqref{con:4} ensures that all vehicles start from the depot.

This basic formulation should be augmented with a few extra constraints that fix some variable values.
We make the assumption that all vehicles start from the depot;
when combined with the constraints in the base formulation, this implies that some variables must be zero:
\begin{align}
&\label{con:5}  x_{v,0,i} = 0, \quad \forall v \in V, i \in N, \\
&\label{con:6}  x_{v,1,j} = 0, \quad \forall v \in V, (d,j) \notin E.
\end{align}
%Constraint~\eqref{con:4} ensures that each vehicle starts at the depot.
Constraints~\eqref{con:2} and \eqref{con:4} together imply Constraint~\eqref{con:5},
while Constraints~\eqref{con:3} and \eqref{con:4} together imply Constraint~\eqref{con:6}.

This formulation has bilinearities in the objective and equality constraints.
Reformulation with linear inequalities is possible to get a mixed-integer linear program.
We will instead derive an equivalent quadratic unconstrained binary optimization (QUBO) problem.


\section{QUBO reformulation}
Expressing the linear equality constraints of the formulation above as $Ax = b$, an exact penalty reformulation is
\begin{equation}
\label{qubo}
\begin{aligned}
\min_x\; & \sum_v \sum_p \sum_{(i,j) \in E} c_{i,j} x_{v,p,i} x_{v,p+1,j}  + \rho\norm{Ax - b}^2 +  \rho\sum_v \sum_p \sum_{(i,j) \notin E}  x_{v,p,i} x_{v,p+1,j} \\
\st
& \eqref{con:4}, \eqref{con:5}, \eqref{con:6}, \\
& x_{v,p,i} \in \set{0,1}, \quad \forall (v,p,i).
\end{aligned}
\end{equation}
For penalty parameter $\rho$ sufficiently large, the solutions of the two math programs coincide.
Looking at the specifics of the constraints and considering that the variables are binary, the smallest value that the penalty terms can take for an infeasible solution is $\rho$ ($=\rho\cdot 1$).
Thus, $\rho$ needs to be big enough to overwhelm any decrease in the original objective by moving to an infeasible point.
Imagine flipping each bilinear term from $0$ to $1$ or vice versa depending on the sign of $c_{i,j}$;
we can upper bound that change in objective by
$\sum_v \sum_p \sum_{(i,j) \in E} \abs{c_{i,j}} = P\card{V} \sum_{(i,j) \in E} \abs{c_{i,j}}$.
Thus 
\[
\rho > P\card{V} \sum_{(i,j) \in E} \abs{c_{i,j}}
\]
suffices.
%
The term $\norm{Ax - b}^2$ expands to $x\tr A\tr A x - 2 b\tr Ax + b\tr b$.
We can add the other bilinear terms from the objective and penalty to get an objective in the form $x\tr Q x$, as desired for a QUBO.


\section{MIRP considerations}
Similar to the path-based formulation, ship (vehicle) capacity determines the demand levels and time windows of the nodes.
Nodes have a fairly narrow time window compared to travel times and the overall time horizon;
we use these time windows and travel times to remove edges from the graph that cannot participate in the solution.
Specifically, if the end of the time window of node $i$ plus the travel time $t_{i,j}$ is greater than the end of the time window of node $j$, then that edge is removed.
Further, we enforce an alternating sequence of supply and demand nodes by pruning out all other edges.
Vehicle starting positions are enforced through dummy nodes and appropriate edges.
A maximum number of sequence positions can be estimated from the shortest travel time and the time horizon.
The reason we consider the depot ``absorbing'' is to allow an overestimate of this number.
%%%%%%%%%%%%%%%%%%%%%%%%%%%%%%%%%%%%%%%%%%%%%
%%%%%%%%%%%%%%%%%%%%%%%%%%%%%%%%%%%%%%%%%%%%%
\bibliographystyle{plain}
%\bibliography{%
./bib/other,%
./bib/books}
%%%%%%%%%%%%%%%%%%%%%%%%%%%%%%%%%%%%%%%%%%%%%
%%%%%%%%%%%%%%%%%%%%%%%%%%%%%%%%%%%%%%%%%%%%%
\end{document}
